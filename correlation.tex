%%%%%%%%%%%%%%%%%%%%%%%%%%%%%%%%%%%%%%%%%%%%%%%%%%%%%%%%%%%%%%%%%%%%%%%%%%%%
%% MH update to correlation_20170306.tex in Bitbucket branch 20170928 
%% VG updates pushed to BB by MH on 2018-02-07
%% MH update to correlationVG2.tex included in folder valeriegares/20160526/correlation_20160522
%% MH variant starting from  Y: _20151209
\documentclass[twoside,a4paper,12pt]{article}

%%%%%%%%%%%%%%%%%%%%%%%%%%%%%%%%%%%%%%%%%%%%%%%%%%%%%%%%%%%%%%%%%%%%%%%%%%%%
%%%%%%%%%%%%package%%%%%%%%%%%%%%%%%%%%%%%%%%%%%%%%%%%%%%%%%%%%%%%%%%%%%%%%%

\usepackage{amsmath,amssymb,amstext,latexsym,amsfonts,graphicx,fancybox}
\usepackage{mathtools}
\usepackage{nicefrac}
\usepackage{amsthm,dsfont,ifthen,mathrsfs,amsfonts,shadethm, nicefrac,tabularx}
\usepackage{amssymb,amsmath,amsthm,amscd}
\usepackage{algorithm,algorithmic}
\usepackage[top=3cm, bottom=3cm,left=2cm,right=2cm,headheight=14.5pt,showframe]{geometry}
\usepackage{inputenc}       %Codage : UTF-8 (saisie des accents)
\usepackage[T1]{fontenc}    %Codage de sortie (accents dans le pdf)
\usepackage{xcolor}
\usepackage{enumerate}  
\usepackage{amsthm,dsfont,ifthen,mathrsfs,amsfonts,shadethm, nicefrac,tabularx}
\usepackage{amssymb,amsmath,amsthm,amscd}
\usepackage{algorithm,algorithmic}
\usepackage{mathrsfs}    
\usepackage[toc,page]{appendix} 
\usepackage{graphicx}
\usepackage{float}
\usepackage{array,multirow}
\usepackage{fancyhdr,fancybox,shadow}
\usepackage[normalem]{ulem}
\usepackage[colorlinks=true,linkcolor=black,citecolor=black,pdfhighlight=/O]{hyperref}
\usepackage{lscape}
\usepackage{natbib}
\usepackage{tikz} 
\usepackage{fancyhdr} 
% MH adjust \headheight:
\setlength{\headheight}{16pt} 
\usepackage{cooltooltips}
\usepackage{authblk}
\pagestyle{fancy}
\fancyhf{}
\fancyhead{}
\fancyhead[C]{}
\cfoot{}
\lfoot{}
\rfoot{}
\fancyhead[LE]{Valerie Gares, Malcolm Hudson, Maurizio Manuguerra, Val Gebski}
\fancyhead[RO]{\leftmark}
\renewcommand{\headrulewidth}{1pt}
\usepackage{lastpage}
% usage: \cventry[spacing]{years}{degree/job title}{institution/employer}{localization}{optionnal: grade/...}{optional: comment/job description}
\cfoot{\thepage\ of \pageref{LastPage}}
\usepackage[width=.75\textwidth]{caption}
\captionsetup{width=12cm}
%%%%%%%%%%%%%%%%%%%%%%%%%%%%%%%%%%%%%%%%%%%%%%%%%%%%%%%%%%%%%%%%%%%%%%%%%%%% 
% Track changes: sourceforge package trackchanges
%\usepackage{fullpage}

%\usepackage[finalnew]{../LatexPackage/trackchanges}
%\usepackage[finalold]{../LatexPackage/trackchanges}
%\usepackage[footnotes]{../LatexPackage/trackchanges}
%\usepackage[inline]{../LatexPackage/trackchanges}
%\usepackage[margins]{../LatexPackage/trackchanges}
% \usepackage{trackchanges}
%\usepackage[margins,movemargins]{trackchanges}
%\usepackage[margins,adjustmargins]{trackchanges}

%\addeditor{MH}
%\addeditor{VG}
%%%%%%%%%%%%% New commands %%%%%%%%%%%%%%%%%%%%%%%%%%%%%%%%%%%%%%%%%%%%%%%%%

%%%%%%%%%%%%%%%%%%%%%%%%%%%%%%%%%%%%%%%%%%%%%%%%%%%%%%%%%%%%%%%%%%%%%%%%%%%%
%%%%%%%%%%%%% New commands %%%%%%%%%%%%%%%%%%%%%%%%%%%%%%%%%%%%%%%%%%%%%%%%%

\newcommand{\R}{\mathbb{R}}
\newcommand{\C}{\mathbb{C}}
\newcommand{\Q}{\mathbb{Q}}
\newcommand{\N}{\mathbb{N}}
\newcommand{\Z}{\mathbb{Z}}
\newcommand{\K}{\mathbb{K}}
\newcommand{\G}{\mathbb{G}}

\newcommand{\M}{\mathbb{M}}
\newcommand{\cL}{\mathcal{L}}
\newcommand{\Hy}{{\mathcal{H}}}
\DeclareMathOperator{\cov}{cov}
\newcommand{\HYPO}[2]{
\begin{cases}
\Hy_0 &:~  #1, \\
\Hy_1 &:~ #2.
\end{cases}}
\def\HR{\operatorname{HR}} 
\def\MR{\operatorname{MR}} 
\def\RD{\operatorname{RD}} 
\newcommand{\EE}{\mathbb{E}}
\newcommand{\VV}{\mathbb{V}}
\newcommand{\Var}{\mathbb{V}}
\newcommand{\PP}{\mathbb{P}}
\newcommand{\D}{\mathbb{D}}
\newcommand{\abs}[1]{\left|#1\right|}
\renewcommand{\hat}{\widehat}
\def\cov{\operatorname{Cov}} 
\def\var{\operatorname{Var}}
\def\corr{\operatorname{Corr}}
\def\det{\operatorname{det}}
\def\SD{\operatorname{SD}}
\def\SE{\operatorname{SE}}
\def\bias{\operatorname{bias}}
\def\BCR{\operatorname{BCR}}
\newcommand{\YY}{Y^{'} Y}
\newcommand{\Ybar}{\bar{Y}}
\newcommand{\XtX}{(X^\prime X)}
\newcommand{\XtY}{X^\prime Y}
\newcommand{\super}[2]{{#1}^{(#2)}}
\newcommand{\Yimp}[1]{Y_{#1}^0}
% \newcommand{\Ym}{Y^{(m)}}
% \newcommand{\Ym1}{Y^{(m+1)}}
% \newcommand{\Bm1}{B^{(m+1)}}
% \newcommand{\Vm1}{V^{(m+1)}}
\newcommand{\Eobs}{\EE^{\text{obs}}}
\newcommand{\Eobsm}{\EE^{0}} %\text{obs},m
\newcommand{\Eobso}{\EE^{\text{obs},0}}

\newcommand{\V}{(Y-X B)^T (Y-X B)}
\newcommand{\Vm}{(Y-X B^{(m)})^T (Y-X B^{(m)})}
\newcommand{\Vhat}{(Y-X \hat{B})^T (Y-X \hat{B})}
\def\VV{\operatorname{V}}
\newcommand{\arhob}{\frac{a-\,\rho \, b}{\sqrt{1-\rho^2}}}
\newcommand{\brhoa}{\frac{b-\,\rho \, a}{\sqrt{1-\rho^2}}}
\newcommand{\ddet}{1-\rho^2}
\newcommand{\sdet}{\sqrt{1-\rho^2}}
\newcommand{\vertchar}[1]{\ooalign{#1\cr\hidewidth$|$\hidewi‌​dth}}
\DeclareMathOperator{\tr}{tr} % bold font
\DeclareMathOperator{\vvec}{vec} % bold font
\DeclareMathOperator{\tensor}{\otimes} % bold font
\newcommand{\dett}[1]{\begin{vmatrix}{#1}\end{vmatrix}}
\newcommand{\del}{\partial}
\theoremstyle{plain}
%\newtheorem{criterion}{Criterion}
\theoremstyle{definition}
%\newtheorem{condition}[theorem]{Condition}
\usepackage[]{graphicx}
\chardef\bslash=`\\ % p. 424, TeXbook
\newcommand{\ntt}{\normalfont\ttfamily}
\newcommand{\cn}[1]{{\protect\ntt\bslash#1}}
\newcommand{\pkg}[1]{{\protect\ntt#1}}
\let\fn\pkg
\let\env\pkg
\let\opt\pkg
\hfuzz1pc % Don't bother to report overfull boxes if overage is < 1pc
\newcommand{\envert}[1]{\left\lvert#1\right\rvert}
\let\abs=\envert

\begin{document}
%\DOIsuffix{bimj.DOIsuffix}
\DOIsuffix{bimj.200100000}
\Volume{52}
\Issue{61}
\Year{2010}
\pagespan{1}{}
\keywords{Bivariate censored regression; Competing risks; Correlation; Time-to-event analysis;   \\
\noindent \hspace*{-4pc} %{\small\it (Up to five keywords are allowed and should be given in alphabetical order. Please capitalize the key
%}\\
\hspace*{-4pc} {\small\it words)}\\[1pc]
\noindent\hspace*{-4.2pc} Supporting Information for this article is available from the author or on the WWW under\break \hspace*{-4pc} \underline{http://dx.doi.org/10.1022/bimj.XXXXXXX} (please delete if not
applicable)
}  %%% semicolon and fullpoint added here for keyword style

\title[Correlated bivariate Normal competing risks ]{Correlated bivariate Normal competing risks: parametric censored regression }
%% Information for the first author.
\author[Val\'{e}rie Gar\`{e}s {\it{et al.}}]{Val\'{e}rie Gar\`{e}s\footnote{Corresponding author: {\sf{e-mail: valerie.gares@insa-rennes.fr}}, Phone: +332 23 23 89 48}\inst{,1}} 
\address[\inst{1}]{Univ Rennes, INSA, CNRS, IRMAR - UMR 6625, 20 Avenue des Buttes de Coësmes CS 70839, 35708 Rennes cedex 7}
%%%%    Information for the second author
\author[]{Malcolm Hudson\inst{2,3}}
\address[\inst{2}]{NHMRC Clinical Trials Centre | University of Sydney | Levels 4-6 Medical Foundation Building | 92-94 Parramatta Rd | Camperdown NSW 2050 | Australia}
\address[\inst{3}]{Department of Statistics | Macquarie University | Balaclava Road | North Ryde | NSW | 2109 | Australia}
%%%%    Information for the third author
\author[]{Maurizio Manuguerra\inst{3} }
\author[]{Val Gebski\inst{2} }

%%%%    \dedicatory{This is a dedicatory.}
\Receiveddate{zzz} \Reviseddate{zzz} \Accepteddate{zzz} 

\begin{abstract}

In problems with a time to event outcome, competing risks may arise when 
subjects experience competing events which prevent the outcome of interest in 
the study occurring. 
Methods generalizing the logrank test for proportional hazards models are 
commonly used to examine group differences account for the competing risks. 
Knowledge of the level of correlation between the different risks can assist in 
the understanding and interpretation of the statistical results of subsequent 
analysis. 
 We examine a bivariate normal censored (BNC) linear model using imputation to estimate the 
correlation between times to two competing events
for which the first observed precludes observing the other. 
This approach allows us to impute the survival time for the censored outcome of 
interest when censoring resulted from the occurrence of a competing event. 
Our R-package bnc provides maximum penalised likelihood parameter estimation using an EM algorithm.
 Including a covariate indicating treatment group  in the bnc linear model provides an 
estimator of relative increase in survival for a treatment versus a control. 
Using bnc, we study bias/variance properties of such MPL estimation. 
We present simulation results for one-sample and two-sample survival comparisons of time to an event of interest, 
with independent censoring and additional censoring by a correlated competing risk. 
Key parameters, mean, hazard ratios, and correlation are estimated. 
The methods were applied to a trial of head and neck cancer. 
 This approach helps to guide both understanding and interpretation of other 
standard methods for these problems. 
\\
\end{abstract}

\noindent \textbf{Key-words:} Time-to-event analysis, competing risks, 
correlation, bivariate censored regression, hazard ratio.\\



%% maketitle must follow the abstract.
\maketitle                   % Produces the title.

%% If there is not enough space inside the running head
%% for all authors including the title you may provide
%% the leftmark in one of the following three forms:

%% \renewcommand{\leftmark}
%% {First Author: A Short Title}

%% \renewcommand{\leftmark}
%% {First Author and Second Author: A Short Title}

%% \renewcommand{\leftmark}
%% {First Author et al.: A Short Title}

%% \tableofcontents  % Produces the table of contents.


\section{Introduction}

Over the past three decades there have been major methodological advances in 
addressing time to event problems. 
The logrank test, which tests the equality of survival distributions from 
censored event time data, \citep{Mantel1959} and the proportional hazards model 
are commonly used in the analysis of cancer data \citep{Cox1972}. 
The accelerated failure time is a useful alternative to the proportional hazard 
model in survival analysis \citep{Wei1992}. 
Methods for competing risks data include cause-specific (CS) proportional hazard 
models \citep{Prentice1978}. 
Fine and Gray have provided a class of proportional subdistribution hazard (SD) 
models commonly used to account for competing risks \citep{Gray1988,Fine1999}. 
Both models can contribute to analysis and interpretation of survival data with 
competing risks  \citep{Dignam2008,Dignam2012}.
Both CS and SD hazard approaches generalize the logrank test and Cox partial 
likelihood tests for a univariate survival outcome. 
\\

An important factor affecting the understanding and choice of time to event 
analysis is the strength of the correlation between the different risks. 
\emph{Our aim is to assess the effects of correlation between two competing 
risks on the estimator of the ratio of the mean} times to event for a 
treatment versus a control. \\


The logarithm of the time to event may satisfy assumptions of normality as used 
in the approach of Royston \citep{Royston2006}. 
For a single survival outcome, a normal censored linear regression with failure 
censored data using an iterative least square method is implemented by Schmee 
\citep{Schmee1979}. 
The Expectation-Maximization (EM) algorithm \citep{Dempster1977} reduces bias in 
estimating error variance \citep{Aitkin1981}.
For multivariate survival data, authors have proposed parametric, 
semi-parametric and non-parametric estimation \citep{Hougaard}. 
The parametric models permit extrapolation of long-term event probabilities, 
which are of inherent interest and which cannot generally be identified from nonparametric and semiparametric models. 
Moreover, parametric regression models are amenable to formal maximum likelihood (and Bayesian) inferences
Parametric methods may be preceded by logarithmic transformation of the failure time.
This provides an accelerated failure time (AFT) model, which simplifies interpretation 
of regression models for cause-specific mean times to failure 
over that provided in semi-parametric models of effects on the cumulative incidence function such as the model of \cite{Fine1999}.
A parametric model for competing risks using the Gompertz (1825) distribution to parameterize the log baseline cumulative
subdistribution hazard function is described in \citep{Jeong2007}.
% not sure to fit in table 3 doesn't work better at each time...
Alternatively, the Buckley-James method \citep{Buckley1979,Lin1992,Jin2006} 
provides a semi-parametric model. The method is based on a EM algorithm and %non-parametric models using 
linear rank statistics \citep{Louis1981}.
The Buckley-James method and its extensions \citep{Lin1992} ignore the dependence 
structure in the data and correlation in multivariate failure times.  
Methods have been developed  to estimate the correlation of multivariate failure 
times using multiple imputation \citep{Pan1999}. 
This correlation is also considered in \citep{Jin2006}. 
In developing methods for competing risks analysis, 
we consider models for competing risks in which both observations and model form differs, 
%\annote{univariate}{the last two sentences were about multivariate correlation models. 
% What distinguishes our approach? \\
% [VG] [33] and [19] non parametric models and in both case we observe for one subject 2 survival times [33] 
% Bivariate survival data  in twin studies and studies of both eyes or ears of the same individual, correlation between the survival times, 
% NOT THE SAME PROBLEM NOT THE SAME CORRELATION, [19] consider the correlation structure of multivariate failure times 
% when a subject can potentially experience several types of event or failure (WE OBSERVE SEVERAL SURVIVAL TIME FOR ONE SUBJECT).} 
in that %marginal univariate models are fitted to 
data on each of the bivariate response times is not available for competing risks.
Instead the first occuring event time of correlated competing risks is observed (together with event cause. \\

There is well known non-identifiability for models which specify only marginals \cite{Crowder1991}. 
Censored regression model (CRM) approaches using a specified parametric 
distribution may avoid ill-posedness in the problem formulation.
While the bivariate distribution assumed is not verifiable from observed data, 
its specification allows us to assess 
sensitivity of the estimator of treatment benefit to correlation. %the ratio of means. \\

Our approach provides an EM algorithm for competing risks (generalising one for univariate survival  (\cite{Aitkin1981})) which provides a maximum penalised likelihood (MPL) solution. Complex calculations are evaluated in closed form using a lemma of Stein.

An EM algorithm is used to iterate between imputation of moments of censored failure 
times for which censoring was due to the occurrence of a competing event, loss 
to follow up or termination of the study  (E-step)
and maximum likelihood estimation (MLE) with imputed moments (M-step). 

Including a covariate  indicating treatment group  in this BNC model 
provides an estimator of $\MR$, the ratio of means, for a treatment versus a control. 

Two strategies were used to impute bivariate survivals in the E step of the EM 
algorithm. 
Firstly, using  a prespecified correlation we investigate sensitivity of an  
estimator of the mean ratio ($\MR$) to the correlation input. 
Secondly, the correlation  may itself be estimated in the M-step. 

Standard semi-parametric methods, 
proportional cause-specific (CS) and sub-distribution (SD) hazard models) 
used to assess treatment benefit are applied 
and compared with the proposed parametric method. 
\\


In time to event problems in head and neck cancer, the problem of competing risk 
is common. 
A non-randomized study  documented prospectively a case series 
of patients receiving surgery
with histologically positive nodes in the neck or parotid \citep{OBrien1997} . 
The trial compared a group of patients  receiving adjuvant radiotherapy (XRT) 
with the remaining patients who were not assigned XRT. 
The two competing risks appropriate to the study objective are "local regional 
relapse only" or other cause event (distant relapse, intercurrent death, 
\emph{etc}). 
We compared the standard survival methods for competing risks with the method 
described above.\\

The paper is organized as follows. 
In Section \ref{model}, we recall some useful background on survival analysis 
with competing risk and introduce the BCR model. 
In Section \ref{model}, we describe simulation studies 
conducted to assess the performance of the BCR model:
  to evaluate the performance of our EM algorithms in Section  \ref{simul1}; 
  and
  to assess the effect of the correlation 
    on the estimation of relative increase in survival % under bivariate log-normal assumptions
    and
  to compare the quality of estimates using the different survival 
    {models} depending on this correlation in  Section \ref{simul2} . 
We applied this method in a head and neck cancer trial in Section  \ref{ex}. 
A summary and research questions for future  work conclude the paper in Section \ref{summary}.

\section{Censored linear models of survival}\label{model}

\subsection{Notation}

In this section, we establish notation and recall different survival 
models.\\

\subsubsection{Univariate survival analysis}
Let $T$ be a non-negative random variable with cumulative distribution function 
$F$, survival function $S=1-F$ and hazard function  $\lambda \, dt =dF/(1-F)$. 
$T$ denotes the duration between the time origin and the time of occurrence of 
some event of interest. 
The survival $Y= \log T$ is assumed to be right-censored: we observe the event only if it occurs 
before a specified time, i.e. only if  $Y < C$. 
Censoring time $C$ %has survival function $S_C$ and 
is assumed to be independent of $Y$. 
{We} observe $n$ independent subjects; the {$j$-th} subject
has survival and censoring times $Y_j$ and $C_j$ 
respectively. 
The observations consist of the $n$ couples
$(y_j,\Delta_j), \quad {j=1\ldots n}$, where $y_j=\min(Y_j, C_j)$, $\Delta_j=\mathbb I\{Y_j\leq 
C_j\}$ and $\mathbb I$ is the indicator function. 
Cox proposed a proportional hazards (PH) model \citep{Cox1972} which relates the 
hazard function at time $t$, $\lambda(t,X_j)$, and the covariates (row) vector 
$X_j$ and  for the $j$-th subject, by:
$$
\lambda(t, \,X_j)=\lambda_{0}(t)\exp( X_j \beta),
$$
where $\lambda_{0}$ is the baseline hazard function and $\beta$ is an unknown 
coefficient vector. 
Alternatively, we may consider an AFT model in which:
$$
Y_j =X_j \beta+\epsilon_j,
$$
where $\beta$ is a vector of unknown regression parameters and 
$(\epsilon_i)_{j=1,\ldots, n}$ are independent error terms with a common 
distribution chosen bivariate-normal or otherwise. 
\\

\subsubsection{Competing risks}


A competing risk is an event whose occurrence precludes the occurrence of the primary event of interest
% an event that either hinders the observation of the event of interest or modifies the chance that this event occurs.
As our interest is in time to occurrence of one specific type of event, %(without the intervention of a competing event), 
all others events can be grouped together as a common event type; without loss of generality we assume K=2.
It is convenient in our model to measure times on a log-scale,
so we define $Y=(Y_1, Y_2)$ to be the logs of the latent times of events of type 1 and 2, 
both subject to censoring at log-time $C$. 
For the $j$-th subject, let:
\begin{align*} % define delta
\delta_j&=
\left\{
\begin{aligned}
1&:~\text{if an event is observed before end of follow-up} \\ % event 1 -- the event of interest -- is first to occur
0&:~\text{if no event occurs.}
\end{aligned}
\right.
\end{align*}
{Both} {event types} are subject to the same {independent} censoring,
{but only the first event is observed}. 
Therefore, we observe $(y_j,D_j=\Delta_j\delta_j)_{j=1\ldots n}$, with 
{$y_j=\min(Y_{1j},Y_{2j},C_j)$,}
$\Delta_j$ indicating absence of {independent} censoring. Here
\begin{align*}
D_j&=
\left\{
\begin{aligned}
1&:~\text{if the event of interest is observed,} \\
2&:~\text{if an event of a competing risk is observed,} \\
0&:~\text{if no event is observed during follow-up}.\\
\end{aligned}
\right.
\end{align*}

For event cause 1, let the cumulative incidence function $F_1(t)=\PP(Y_1\leq t,\delta=1)$
and the subdistribution hazard function 
$\lambda_1(t)\,dt=dF_1(t)/[1-F_1(t)]$. 
Fine and Gray proposed a proportional sub-distribution (SD) hazards model 
\citep{FineGray1999} which specifies the sub-distribution hazard function {$\lambda_1(t,Z_j)$}  
for the cause of interest %\change[MH]{$h_1(t,Z_i)$}
%\note[MH]{Its *not* the CS hazard, its definition differs. I am adopting the notation in a recent book on CR by Geskus. Otherwise, you could write $\lambda_{1,SD}$ or similar} 
of subject $j$, with covariates $X_j$, as:
$$
{\lambda}_1(t \, | \, X_j)=\lambda_{1}(t|0) \, \exp(X_j \beta_1),
$$
where $\lambda_{1}(t|0)$ is the baseline SD-hazard function and $\beta_1$ a vector of coefficients.
%\change
%{\subsection{Censored regression with bivariate normal competing risks}}
\subsection{The BNC linear model: bivariate normal competing risks}

In this section, we assume a bivariate normal distribution of survival 
components to assess the correlation between two competing risks.
Specifically, we assume that the vector of logarithms of the time to the event 
of interest and to an event of another cause, $Y=(Y_1,Y_2)$, is bivariate normally 
distributed with correlation $\rho$. 
\\

Again denote by $Y_1$  the logarithm of time to event 1 and by $Y_2$ the logarithm of 
time to {event 2}. 
Assume $Y=(Y_1,Y_2)$ follows a bivariate normal distribution with specified means and covariance matrix $\Sigma$, $2$x$2$,
%\remove[MH]{with $\EE(Y_k)=\mu_k$, $\var(Y_k)=\sigma_k^2$, for $k=1,2$, }
with $\corr(Y_1,Y_2)=\rho$. 
Let $C$ denote the logarithm of the time of censoring.
As before, let $y= \min(Y_1,Y_2, C)$.
The observed data is a random sample consisting of $n$ pairs
$(y_j,D_j)$, for ${j=1\ldots n}$,  where 
$y_j=\min(Y_{1j},Y_{2j},C_j)$, and $D_j \in \{0,1,2\}$ {is defined above}. 
\\
% \remove{For the $j$-th subject and the event $j$, to relate} 
The expected survival $\EE Y$  
{depends on subject specific} covariates $X$ 
{with $\EE Y = \mu= X B$, where $B =(\beta_1,\beta_2)$ is a matrix of}
{cause-specific regression coefficients of dimension $p~\text{x}~2$}. 
The BNC linear model specifies the means for subject $j$ as $(\mu_{1j},\mu_{2j}) =X_j B$.




\subsection{Censoring status and Likelihood function}

In this section we set notation for event and censoring status. 
We then define 
likelihood functions for competing risks data. %first-event and for full

In competing risks data all observations are subject to censoring,
not only by end of follow-up, but also by competing events.
With an event of cause 1, $D_j=1$,   time to event 2 is censored; 
i.e. $Y_{1j}=y_j$ is observed and $Y_{2j} > y_j$ is censored. 
Similarly, when $D_j=2$  $Y_{2j}= y_j$ is observed and 
$Y_{1j}> y_j$. 
Finally, $D_i=0$  when times to both events are censored; 
for observed censoring time $y_j$,  $Y_{ij}> y_j$ for $i=1,2$. 
\\

% The {likelihood} function for data  on time to first event ($Y_1$), with independent 
% censoring by competing risk or end-of-study, is 
% %\begin{footnotesize}
% \begin{align}\label{lik1} % [MH]
% L_1 &=&\prod_{j:\delta_j=1}f_{Y_1}(y_j) 
% \prod_{j:\delta_j = 0 } P \{ Y_1 > y_j \},% \prod_{j:\D_i=0} S_{Y_1}(y_j),
% \end{align}
% %\end{footnotesize}
% where (?) \ldots% $S_{Y_1(t)}$ denotes the survival function of the event cause of interest (cause 1).%, the probability time to first event exceeds $t$.

The {likelihood function} for fully observed competing risks data observations 
$(y_j, D_j; \; j=1,\ldots,n)$ is:
%$M=\{y_j, \Delta_i \delta_i; j = 1, \ldots, n\}$  is written:
{\small
\begin{equation}\label{lik2}
L_{y,D}(B, \Sigma) = \prod_{j: D_j=1} f_{Y_1}(y_j) \, F_{2|1}(y_j|y_j) %P \{ Y_2> y_j \,| Y_1=y_j \}
    \prod_{j: D_j=2} f_{Y_2}(y_j) \, F_{1|2}(y_j|y_j) %P\{Y_1 \, | Y_2 = y_j\}
    \prod_{j: D_j=0} F_{12}(y_j, y_j), %P \{ Y_1 > y_j, \, Y_2 > y_j \},
\end{equation}
where $F_{1|2}(a|b) =P\{Y_1 > a \,|\, {Y_2 = b\}}$, $F_{2|1}(a|b) =P\{Y_2 > a \,|\, {Y_1 = b\}}$ and $F_{12}(a,b)= {P\{Y_1>a,} \, {Y_2>b\},}$ %$S$ denotes a survival time, and $S(t,t)$ is the probability of (log-)times to events of both types exceeding $t$.
with probabilities depending on the coefficient matrix $B$ and covariance matrix $\Sigma$ of the bivariate Normal distribution.
}

Let $\mu_1$ and $\mu_2$ be the columns for cause 1 and cause 2 of the matrix of expected values $\mu=X B$ of $Y=(Y_1,Y_2)$.  % $\EE (Y_{1j})=\mu_{1j}$ and $\EE Y_{2j} = \mu_{2j}$, 
Define $z_{i}= (y -\mu_{i})/\sigma_i$,  %$\bigl(\begin{smallmatrix} \sigma_1^{-1}&0 \\ 0&d\sigma_2^{-1} \end{smallmatrix} \bigr)$
with $ \sigma_i = \sqrt{\sigma_{ii}}$,
for $i=1,2$.
Then $(Y-\mu) W   \sim \text{BVN}(0, R)$ where the $2$x$2$ weight matrix $W$ is diagonal with entries $\sigma_1^{-1}, \sigma_2^{-1}$,
and 
% 
% \[
%   R= \begin{bmatrix}
%   1 &\rho \\
%   \rho & 1
%  \end{bmatrix}
% \]
% or,
$R= \bigl(\begin{smallmatrix} 1& \rho \\ \rho & 1 \end{smallmatrix} \bigr)$
Thus, given the covariate matrix $X$, all likelihood terms may be expressed in terms of the standard bivariate normal distribution;
the likelihood is a function of $\rho$ and values $z_1, z_2, D$ of the observed random sample.


\subsubsection{Expectation Maximization algorithm}

An EM algorithm {at each iteration imputes} missing data (censored data here), 
{substituting} the expectations {of sufficient statistics} conditional on the observed data and the 
current parameter estimates (E step). 
The new parameter estimates are obtained from {these imputed} sufficient statistics as 
though they had come from a complete sample (M step). 
\\


The EM algorithm for likelihood function of survival $Y_1$ %in equation \eqref{lik1} 
not subject to competing risks is described in \cite{Lunn1998}. 
For likelihood function for observation $(Y_1,Y_2)$ in equation \eqref{lik2}, 
the parameters at each step $r$ consist of 
$\theta_{(r)}= \left( B^{(r)}, \Sigma^{(r)} \right)$. %\left(\mu^{(r)}%(\mu_{1,(r)},\sigma_{1,(r)},\mu_{2,(r)},\sigma_{2,(r)}, \rho_{(r)})$ 
The new estimate $B^{(r+1)}$ of $B$ %\mu_{1,(r+1)} \mu_1
is obtained as $\hat{B}=\XtX^{-1}X^\prime Y$ with imputed $Y$.
\remove[VG]{by replacing the} 
For this estimation, when $D_j=2$, so that $Y_{j1}$ is censored by the event of cause 2 at $y_j$,  
the censored observation $Y_{j1}$ is imputed by 
$\EE(Y_1|Y_1>y_j,\,Y_2=y_j,\;\mu_j^{(r)},\Sigma^{(r)})$, 
when $D_j=0$,  by 
$\EE(Y_1|{Y_1>y_j,}\,{Y_2>y_j;}\; \mu_j^{(r)},\Sigma^{(r)})$, where $\mu=XB$. 
A complete data sufficient statistic for the covariance matrix $\Sigma$
is $V= (Y-X\hat{B})^{'} (Y-X\hat{B})=Y^\prime Q Y$ for known projection matrix $Q=I-X \XtX^{-1} X^\prime$.
The EM update to $\Sigma$ therefore includes imputation of quadratic terms (squares and cross-products) in $(Y_1,Y_2)$.
The new estimate %$\sigma_{1,(r+1)}$ of $\sigma_1$ 
is obtained using the censored observation $y_j$ 
by replacing linear terms as above and 
%the censored $M_i^2$,  
quadratic terms using appropriate conditional distributions.
For $j \neq j^\prime$ the statistical independence of observations reduces calculations to imputation of linear statistics,
but for a single observation $j$ complex conditional expectations must be evaluated.
For example, when 
$D_j=2$,  a quadratic term $Y_{j1}^2$ in the censored observation $Y_{j1}$ 
is imputed using $\EE(Y_1^2|{Y_1>y_j,}\,{Y_2=y_j}; \; \mu_j^{(r)},\Sigma^{(r)} )$.
Similarly, when $D_j=0$,  the same term is imputed as 
$\EE(Y_1^2|{Y_1>y_j,}\,{Y_2>y_j;} \; \mu_j^{(r)},\Sigma^{(r)}))$. 
%Similar methods give the estimates $\mu_{2,(r+1)}$ and $\sigma_{2,(r+1)}$. 
%The new estimate $\rho_{(r+1)}$ of $\rho$ is obtained by replacing the censored 
%$y_j$ and $y_j^2$ as above and 
%the product $y_{1,i}y_{2,i}$,  
When 
$D_j=0$,  we require $\EE(Y_1 Y_2|{Y_1>y_j,}\,{Y_2>y_j;} \; \mu_j^{(r)},\Sigma^{(r)}))$. 
We provide all required probabily results for the EM algorithm in a bivariate normal censored linear model in Appendix \ref{appendix:prob}.
These results for the bivariate normal distribution appear to be new, and may prove useful in other contexts. 

We stop the EM algorithm using the Euclidean distance between parameter 
estimates at the previous and the current step of the algorithm 
$\left\|\theta_{(r+1)}-\theta_{(r)}\right\|^2/\left\|\theta_{(r)}\right\|^2< 
\epsilon$ where $\left\|\cdot\right\|^2$ denotes the usual Euclidian norm and 
$\epsilon$, a positive tolerance for convergence. 

Standard errors of the EM algorithm solution  \emph{can be} obtained using the DCM method of Jamshidian and Jennrich \cite{Jamshidian2000}.
The score statistic used is derived from evaluations of the score function $Q(\theta,\tilde{\theta})=\EE[\log L(Y; \tilde{\mu},\tilde{\Sigma})| y, D, \mu, \Sigma]$, 
which is accessible for computation using our probability results.

Complementary details are reported in the Appendix.% \ref{annex}.

\subsection{Mildly Penalize Likelihood}

As the convergence can be to a boundary $\hat{\rho}_{\mbox{ML}} = +1$ or $-1$ and the  convergence can be very long, we use a mildly
penalize likelihood (MPL) (\cite{Barnard2000,CHEN20091367}) which has  the form:
$$
L^p_{y,D}(B, \Sigma) = L_{y,D}(B, \Sigma) + p_n( \Sigma)
$$
where $p_n(\Sigma)$ is the penalty depending on $\Sigma$ and the
sample size $n$.  We consider a family of Wishart (not inverse-Wishart) densities for the prior on
$\Sigma$. The Wishart density function on $\Sigma$ with hyperparameters $\nu$ and $\Psi$ defined by:
$$
p_n( \Sigma)=\frac{|\Sigma|^{(\nu-d-1)/2}\exp(-1/2tr(\Psi^{-1}\Sigma)}{2^{\nu d /2}|\Psi|^{\mu/2}\Gamma_d(\mu/2)}
$$
where $\Gamma_d(\mu/2)}=\pi^{d(d-1)/4}\prod_{j=1}^d \Gamma(\nu/2 +(1-j)/2 $, $\nu$ is the degrees of freedom,
and $\Psi$ is a scale matrix with $\EE[\Sigma]=\mu\Psi$.
We then adjust the EM algorithm for this MPL. This ensures the convergence. Finally, as the convergence was still very slow, we use an accelerate EM (R package {\tt{turboEM}} (\cite{Bobb2014})). This algorithm is implemented  in a R package named {\tt{bnc}}.

\section{Simulation studies } 
\subsection{Performance of the EM algorithm in the BCR model}\label{simul1}

In this section, we study the effect of changes in parameters on the performance of maximum likelihood (ML) estimators -- obtained using the EM algorithm -- of 
the parameters $\theta=(\mu_{1},\sigma_{1}^2, \mu_{2},\sigma_{2}^2, \rho)$ in the BCR model. 
We do not consider covariates. Performance criteria for all parameter estimates are bias and standard error in N=1000 simulations.\\

The bias of an estimator $\hat{E}$ of $E$ is defined as: $\bias(\hat{E}) 
=\EE(\hat{E})-E$. We show its estimate, using the average of the sampling distibution of $\EE(\hat{E})-E$.
The notation $\SE(\hat{E})$ corresponds to the standard error of the estimator $\hat{E}$.
This is the sample standard deviation of $\hat{E}$ in N=1000 simulations.
%\note[MH]{An alternative is to calculate the true SD from the asymptotic variance Var($\hat{E}|E)$. 
%As you know, this variance is that of the MLE and is obtained from the Fisher information matrix for BNC likelihood.}
\\
%Let the Fisher transformation of correlation coefficient $\rho$, $\rho^*=(1 / 2)\log\left((1+\rho)/(1-\rho)\right)$.

\subsubsection{Simulation design.} 
The latent survival times $(Y_1,Y_2)$ are bivariate normally 
distributed with parameters $(\mu_{1},\sigma_{1}^2,\mu_{2},$ $\sigma_{2}^2,\rho)$. 
We have chosen $\mu_1=0$ (so median time to the event of interest is 1 on the original scale), 
$\sigma_1=1$, $\sigma_2=1$. 
Here $\tau$ denotes the maximum length of follow-up and is chosen so that 
a pre-specified proportion, $S_{Y_1}(\tau)$, of latent events of cause 1 were yet to occur. 
$\epsilon$ is chosen to be $10^{-4}$ in EM algorithm. 
\\

We consider four simulation scenarios varying $\rho$, $\mu_{2}$, $n$ and 
$S_{Y_1}(\tau)$. 
\begin{enumerate}[label=(\roman*)]
\item To investigate the effect of correlation on the bias and standard  errors,we consider  correlation $\rho \in \{-0.50, -0.25, 0, 0.25, 0.50\}$.
\item  To investigate the effect of the censoring proportion, we select $\tau$ for values of $S_{(Y_1,Y_2)}(\tau)$ in $\{0, 0.2, 0.4\}$. In the particular cas where $\mu_1=\mu_2=0$ and $\rho=0$ $\tau$ in $\{+\infty, 1.25, 0.75\}$ ($\{+\infty, 3.49, 2.12\}$ on the original time scale) respectively). 
\item To investigate the effect of sample size $n$, we consider $n \in \{50, 250, 1000\}$
\item  To investigate the effect of the difference in means (Table 1d), we take $\mu_{2}$ 
equal to $\{0, 0.25, 0.5, 1\}$ (from $1$ to $2.72$ on the original time scale)
\end{enumerate}
For each configuration of these parameters, $M=1000$ data sets are simulated.

Firstly, in order to investigate the effect of correlation on the bias and standard 
errors (refer Fig 1a), we consider  correlation $\rho \in \{-0.2, 0, 0.2, 0.4, 0.6, 
0.8\}$, for $\mu_{2}=1$, $n=1000$ and $\tau$ such that $S_{Y_1}(\tau)=0.2$. 

Secondly, to investigate the effect of the censoring proportion 
we select $\tau$ for values of $S_{Y_1}(\tau)$ in $\{0, 0.2, 0.4\}$, 
for fixed $\rho=0.2$, $n=1000$ and $\mu_{2}=1$. 
With $\mu_1=\mu_2$ and $\rho=0.2$, the corresponding probabilities that first event occurs both before time $\tau$ and 
before the second event  are 0.50, 0.47, 0.41, respectively. 
Thirdly, to investigate the effect of sample size $n$ (refer Table 1c), we consider $n \in \{200, 400, 600, 800, 1000\}$, $\mu_{2}=1$, $\rho=0.2$ and  $S_{Y_1}(\tau)=0.2$

Finally, to investigate the effect of the difference in means (Table 1d), we take $\mu_{2}$ 
equal to $\{0.5, 0.75, 1, 1.5\}$, for $\rho=0.2$ and $S_{Y_1}(\tau)=0.2$. \\

In the example below showing one simulation, we have chosen $n=1000$, 
$S_{Y_1}(\tau)=0.2$, $\rho=0.25$ and $\mu_2=0$.\\

\noindent \textit{Figure \ref{simuldist}}\\ % Fig.1
\textit{Figure \ref{simulconv}}\\ % Fig 2

Figure \ref{simuldist} provides the boxplot of the simulated data $M$ according 
to the different values of $\Delta\delta$. 
%Since both event types have identical marginal distributions, the upper two 
%boxplots are very similar, with means near 0.5, below the  mean values, both 1%
%, of the unselected survival times. 
%This is a consequence of the \emph{smaller} of the event times being observed. 
%Censoring  due to the end of the study was imposed at a fixed time, as observed 
%in the lower boxplot.\\

Figure \ref{simulconv} illustrates the convergence of means, variances, $\rho$ 
and log likelihood using the BCR model. 
%It shows good estimations of the means and variances but a flat likelihood  and 
%poorly conditioned estimator of $\rho$. Small changes in data will have a strong 
%influence on this estimate.\\

Figure \ref{simuldistz} %Fig. 3
displays the boxplot of $\hat{\rho}$ in the particular case when 
$S_{(Y_1,Y_2)}(\tau)=0.2$, $\rho=0.25$ and $\mu_2=1$. 
%The estimator has  bias close to $0.2$, 
%however the variance is substantial. 
%The majority of values are less than $0.2$ again showing an underestimation of 
$\rho$.\\

Figure \ref{simulest} provides plots of the mean and standard errors in simulations estimates of 
$\rho$ as a function of each parameter, given $\mu_{2}=0$ and 
$S_{(Y_1,Y_2)}(\tau)=0.2$, and confirms all the comments described above. 
%We note a small and positive bias for negative values of $\rho$ and a negative 
%and larger bias for positive values of $\rho$.\\


\noindent \textit{Table \ref{simul3}} \\
%\noindent \textit{Table \ref{simulf}} \\
\textit{Figure \ref{simuldistz}}\\
\textit{Figure \ref{simulest}} \\


\subsubsection{Simulation Results}
Table \ref{simul3} % Table 1
shows ML estimates of  correlation $\rho$.  Estimates of the means and variances are presented in the suplementary materials.

%Firstly (see suplementary materials), estimates of means and variances,  for fixed $\mu_1=\mu_2=0$, $S_{T_1}(\tau)=0.2$ and $n=1000$,
%have a small 
%positive bias as $\rho$ increases. This bias first increases then decreases with increasing $\rho$. The ML estimator of $\rho$ exhibits a small negative bias.
%ML estimates of  $\rho$ show a negative bias. 
%The bias is small when $\rho$ is less than $0.2$. 
%The negative bias implies underestimation of the correlation. 
%
%Secondly (Table 1b), we note a slight decreasing bias of $\mu_1$, $\sigma_1$ and $\rho$, when the 
%proportion of observations with first-event 1 increases,
%as $\mu_2$ increases with $\mu_2\leq \mu_1$. 
%Also note an slightly increasing bias of $\mu_2$ and $\sigma_2$. 
%Corresponding results are obtained when $\mu_2\geq \mu_1$. 
%
%
%Thirdly (Table 1c), estimates of means, variances  generally exhibit a  positive and increasing -- but small -- bias 
%as $S_{Y_1}(\tau)$ (the proportion of events of type 1 beyond time $\tau$) increases.
%
%Finally (Table 1d),
%the bias decreases as sample size $n$ increases,
%for fixed $\rho=0.2$, $\mu_1=\mu_2=1$ and $S_{T_1}(\tau)=0.2$ .

Secondly (Table 1b), we note a slight decreasing bias of $\mu_1$, $\sigma_1$ and $\rho$, when the 
proportion of observations with first-event 1 increases,
as $\mu_2$ increases with $\mu_2\leq \mu_1$. 
Also note an slightly increasing bias of $\mu_2$ and $\sigma_2$. 
Corresponding results are obtained when $\mu_2\geq \mu_1$. 

Firstly (see suplementary materials), estimates of means and variances have a small  bias in each case.

Thirdly (Table 1c), estimates of means, variances  generally exhibit a  positive and increasing -- but small -- bias 
as $S_{Y_1}(\tau)$ (the proportion of events of type 1 beyond time $\tau$) increases.

Secondly, the ML estimator of $\rho$, exhibit an increasing  bias as $S_{(Y_1,Y_2)}(\tau)$  increases, as $\mu_2$ increases and as $n$ decreases  (Table 1a, b and c).
%Table \ref{simulf} shows that if one parameter at least is fixed in the model, 

Thirdly, for $ n=1000$ (table 1c), the ML estimator of $\rho$ exhibits a small bias when $\rho\leq 0.25$ inferior to $0.10$. However the variance is substantial.  
%We note a small and positive bias for negative values of $\rho$ and a negative and larger bias for positive values of $\rho$.

Finally, the ML estimator of $\rho$ exhibits a negative bias when $\rho\geq 0.25$, . 
The negative bias implies underestimation of the correlation.\\  For $ n=1000$, we note a small and positive bias for negative values of $\rho$ and a negative 
and larger bias for positive values of $\rho$.\\

%Figure \ref{simuldistz} %Fig. 3
%displays the boxplot of $\hat{\rho}$ in the particular case when 
%$S_{(Y_1,Y_2)}(\tau)=0.2$, $\rho=0.25$ and $\mu_2=0$. 
%The estimator has  bias close to $0.25$, 
%however the variance is substantial. 
%The majority of values are less than $0.25$ again showing an underestimation of 
%$\rho$.\\

Figure \ref{simulest} provides plots of the mean in simulations estimates of 
$\rho$ as a function of each parameter, given $\mu_{2}=0$ and 
$S_{(Y_1,Y_2)}(\tau)=0.2$, and confirms all the comments described above. 
We note a small and positive bias for negative values of $\rho$ and a negative 
and larger bias for positive values of $\rho$.\\
\\

The model demonstrates good estimation of the means, variances and correlation. 
However, it exhibits a small underestimation of the correlation. 
Therefore a sensitivity analysis, fixing the parameter $\rho$ in the model, is 
indicated. 

 \subsubsection{Data with bivariate Weibull distibution}
\noindent \textit{Simulation design.} $(Y_1,Y_2)$ follows a bivariate weibull 
distribution using copulas functions with parameters 
$(\alpha_{1},\lambda_{1},\alpha_{2},\lambda_{2},\theta)$ \citep{Escarela2003}. 
$\theta$ denotes the Kendal correlation.
We have chosen $n=1000$, $\alpha_1=1$, $\lambda_1=0.5$, $\alpha_{2}=1$ and 
$\lambda_2=1$. 
$\tau$ denotes the full study period and is chosen given $1-S_{Y_1}(\tau)$, the 
proportion of events 1 at time point $\tau$. 
Here, we define the censoring time $C^*=\tau$. 
All subjects are followed until the end of the study. 
$S_{Y_1}(\tau)=0$ means $100\%$ of events occur before $\tau$ and $0\%$ remain 
unobserved at $\tau$.  
\\

\subsection{Effects of the correlation on estimation of relative increase in survival}\label{simul2}

In this section, we use an EM algorithm in the BCR model including the 
covariates, to estimate the mean ratio $\MR$ and study its 
performance. 
Varying the correlation allows us to assess its effect on estimation of relative 
increase in survival using the BCR model, and
by standard approaches using proportional CS and SD 
{assumptions}.\\

First, we specify 
an AFT model $\mu_{ij}=\beta_{i,0}+\beta_{i,1}X_{j}$,  for $i=1,2$,
with covariate $X_j$ indicating allocated treatment arm of the $j$-th subject. %and the event $i$. 
For a lognormal distribution, $E(T)=\exp(\mu+\sigma^2/2) $, so the AFT assumption implies
the ratio of mean survivals for cause $i$,  $\MR_i = E(T|X=1)/ E(T|X=0) =\exp \beta_{1,i}$.
Here $1/\MR =\exp(-\beta_{1,i})$ corresponds in values to HR for proportional hazards models.
To estimate $1/\MR$, after log-transforming data 
we  impute bivariate survivals in the E step of the EM algorithm, 
using the ML estimator of the correlation or alternatively with  
prespecified correlation providing a sensitivity analysis. 
%We study the $\RI$ estimators using standard methods. 
\\

The model describes a clinical trial with two arms; we further impose balanced design, equal numbers $n/2$ of patients in 
treatment and control groups. 
We study the estimated hazard ratio (HR) and \annote[VG]{relative increase ($RI=\MR^{-1}$)}{Do we keep this notation $RI=MR^{-1}$? Do we speak about relative increase or ratio of mean survivals?}  for the 
event of interest  for a range of methods
and their respective $95\%$ confidence interval (CI). 
The \change[VG]{$\MR^{-1}$}{RI} ranges in the CI by a factor between $\exp(\pm 1.96 \mbox{SE}(\hat{\beta}_{1,1}))$ around the estimate. 
The estimation methods included are regression coefficient estimates in the CS 
hazard model (in which subjects who experienced the competing risk are 
censored), 
in the SD hazard model (in which subject who experienced the competing risk 
remain in the risk set) 
and in the BCR model using an EM algorithm. 

We will also perform a sensitivity analysis imputing the censored survival times 
of the outcome for different levels of correlation. 
% We cannot compare Fine and Gray and Cox models themselves  because they do not  calculate the same function. 
% Furthermore, these models gives an hazard ratio estimation and these generated data do not respect the assumption of proportional hazard ratio.
% \cite{Kocherginsky2008,James2012}\\
%The hazard ratio of log-normal distribution is represented by figure  when the 
% relative increase is $\exp(0.5)$ and $\sigma=1$.}figure \ref{hazardnp}\\
In what follows, all the random variables and relevant parameters for treatment 
(respectively control) are indexed by \change[VG]{T}{$Z=1$} (respectively \change[VG]{C}{$Z=0$}), and for event 1 
(respectively event 2) are indexed by 1 (respectively 2). 
\\

\subsubsection{Simulation design}




For  $(Y_1^j,Y_2^j)$ is bivariate 
normally distributed with parameters 
$(\mu_{1}^j,\sigma_{1},\mu_{2}^j,\sigma_{2},\rho)$. 
We fix $n=1000$ (i.e. $500$ subjects in each group). 
In both arms,  
%{Under treatment (and under placebo too) $\tau$ is chosen such as $S_{Y_{1}}^{T}(\tau)=0.2$} 
censoring only occurs at end of study and, in the absence of competing risks, the 
proportion of events of interest before this fixed period of follow-up $\tau$, is set so that $S_{Y_{1}}^{Z=0}(\tau)=0.2$. \remove[VG]{** Notation bad: C superscript for control, not C censor time **.}
proportion of events of interest before this fixed period of follow-up $\tau$, is set so that $S_{Y_{1}}^{Z=0}(\tau)=0.2$. 
All variances are chosen equal to 1 ($\sigma_{1}=\sigma_{2}=1$). 

The mean survival time for event of interest (event 1) in the control group 
is chosen equal to 1, $\EE (T | Z=0, \delta=1) =1=\mu_1^{Z=0}$ %($\mu^{C}_{1}=1$). 
The mean for event 1 in the treatment group is then determined by the 
mean ratio $E(T | Z=1,\delta=2) =\mu_1^{Z=0} +\log \MR_1=\mu_1^{Z=0} + \beta_{1,1}$, where  $1/\MR =\exp(-\beta_{1,1})$. 
The competing event (event 2) has the same risk of  occurrence in both arms 
{$\MR_{2}=1$}, $\mu^{Z=0}_{2}=\mu^{Z=1}_{2}=\mu_{2}$) and the latent survival times 
of event 1 and 2  have the same correlation $\rho$ in both group. 
In order to investigate the effect of the correlation and the {relative 
increase}  on the {relative increase} estimation, we consider several simulation 
scenarios obtained by combining various correlation $\rho \in \{-0.25, 0, 0.25, 0.5\}$,  
$\MR_1^{-1}\in \{0.5, 1, 1.5\}$ and 
$\mu_{2}=1$. 
To investigate the effect of higher correlation, we choose $\rho=0.6$, 
$\MR_{1}^{-1}=0.5$ and $\mu_{2}=1$. 
To investigate the effect of the proportion of observed event 1 related of 
observed event 2, i.e. 
the difference in means between the two events, we consider $\mu_{2}\in 
\{1,1.25,1.5\}$, $\rho \in \{0, 0.25\}$ and $\MR_{1}^{-1}=0.5$. 
In each case, we impute bivariate survivals, in the E step of the EM algorithm, 
using the ML estimator of the correlation and we conduct a sensitivity analysis 
imputing bivariate survivals for prespecified correlations $\rho_{\BCR} \in \{0, 
0.25, 0.5\}$ ($\rho_{\BCR} \in \{0, 0.1, 0.2, 0.3, 0.4, 0.5\}$ 
when $\mu_{2}=1$, $\rho \in \{0, 0.2, 0.4\}$ and $\MR_{1}^{-1}=0.5$). 
For each configuration of these parameters, $M=1000$ data sets are simulated.\\


\noindent \textit{Table \ref{simulhr}}\\
% I will calculate again the confidence interval as soos as I am in France. 
% They are wrong: I did, over the $M$ simulations $\frac{1}{M}\sum_{i=1}^M \exp(\beta_i \pm 1.96 SD_i)$. 
% The right computation is $ \exp(\frac{1}{M}\sum_{i=1}^M(\beta_i) \pm 1.96 \frac{1}{M}\sum_{i=1}^MSD_i)$}.

% \noindent \textit{Table \ref{tableex2}}\\

\noindent \textit{Figure \ref{simulhazard}}\\

\subsubsection{Simulation results}
We can note that the lognormal modelling implies a time dependant HR,
so the proportional hazards assumption of CS and SD hazard models are violated 
on these data. 
The Fine and Gray method for a proportional SD hazard model should provide a time-averaged SD HR estimate; 
similarly, a proportional CS hazard model will estimate a time-averaged CS HR \citep{Kalbfleisch1981}.
From Table \ref{simulhr}, it appears that -- with either independent or correlated risks --
the HR estimated by the FG method is closer to $\MR_{1}^{-1}$ than the HR estimated using a proportional CS hazard model, with close agreement when $\MR=1$.  
The difference between the HR estimated by the SD and CS models and 
$\MR_{1}^{-1}$ increases as the correlation $\rho$ increases. 
{The CI of the CS and SD HR estimator does not include the true value $\MR_{1}^{-1}$ 
when $\rho \geq 0.4$.} 
% 
The bias of $\hat{\MR}_{1}^{-1}$, estimated by the BCR model using the EM algorithm 
for a given correlation $\rho_{\BCR}$
is small. 
In each situation, the bias of $\hat{\MR}_{1}^{-1}$, estimated by the BCR model are decreasing near correlation $\rho_{\BCR}$: 
$\hat{\MR}_{1}^{-1}$ is below the true value $\MR_{1}^{-1}$ when $\rho_{\BCR}<\rho$, 
above when $\rho_{\BCR}>\rho$, and almost equal when $\rho_{\BCR}=\rho$. 
The CI obtained with different values $\rho_{\BCR}$ include the true value $\MR_{1}^{-1}$ unless $\rho_{\BCR}$ is far from the true value  $\rho$. 
%
The bias of $\hat{\MR}_{1}^{-1}$ estimated by the BCR model with $\rho_{\BCR}=\rho$ are {small} 
and the confidence 
intervals shortest among all the methods. For all models, as odds of events of type 1 being observed among uncensored observations increases, the bias of $\hat{\MR}_{1}$ increases.
For all models, the true value $\MR_{1}$ influences the standard errors (SEs) of 
the estimator of mean ratio; SE increases when $\MR_{1}^{-1}$ increases. 
 

Figure \ref{simulhazard} displays the hazard ratio or the relative event 
rate and its CI estimated using the different methods (SD hazard model, CS 
hazard model, BNC model) 
in the particular case $\mu_{2}=1$, $\rho \in \{0, 0.2, 0.4\}$ and 
$\MR_{1}^{-1}=0.5$.

\subsubsection{Findings}
% \note[VG]{There was an old table to check the convergence when some parameters were 
% fixed... do you want we introduce ?} \ref{simulhrf}  

These simulations demonstrate the influence the correlation has on estimation of the relative increase and show the robustness of the SD hazard model under  correlated risks {$\rho\leq 0.2$}. 
Sensitivity analysis shows the impact of the chosen correlation on the results. 
Maximum likelihood estimation can lack accuracy.
In such a circumstance, sensitivity analysis is indicated. 
Comparing the different methods can help to determine an upper bound of the 
potential correlation that can be used in prediction.
Another method (see supplement) used to impute the censored times due 
to loss to follow up or end of the study 
did not shown improved estimation of the relative increase.




\section{Example}\label{ex}
  The study compared $n=143$ patients with Head and Neck cancer receiving 
surgery. 
  A treatment group,  $n=45$ patients, also received adjuvant radiotherapy 
(XRT), with control group comprising the remaining $n=107$ patients (no XRT).
  The outcome is time \add[MH]{from surgery} \note[VG]{I checked but it is not written clearly but i guess it is} to first event, the event of interest being local regional 
relapse, other events comprising competing risks. 
The events were recorded as "local regional relapse only" (Event 1, Relapse) or 
"other causes" (Event 2, OC death). 
Patients are censored due to loss to follow up or end of study. 

We use the transformation $x^*:x\to\log(0.25+x)$ for the survival times to provide approximate normality
while allowing for deaths at the time of treatment.\\
  
\noindent \textit{Figure \ref{exdist2}}\\

\nocite{*}

Figure \ref{exdist2} {provides} the box plots of the data.
After the transformation $x^*:x\to\log(0.25+x)$, the survival time marginals 
appear generally consistent with the {bivariate} normality assumption 
{though} more extreme cancer survival times than expected occur for the {outcome 
of interest}. 
In both arms, event 1 more often precedes event 2. 
In both arms and in both events, the variation in %\add{imputed?}\note[VG]{}{It is not imputed yet} 
observed survival times is considerable, particularly for patients not receiving XRT. 
Relapse, Event 1 occurred in 23 patients (3  in the irradiated group and 20 in 
the non-irradiated group), OC death, event 2 occurred in 73 patients  
(23 in the irradiated group and 50 in the non-irradiated group) and 57 patients 
were censored  (16 in the irradiated group and 37 in the non-irradiated group). 
\\

\noindent \textit{Figure \ref{exsurv2}}\\
  
  Figure \ref{exsurv2} provides  plots of cumulative incidence functions in the 
two arms for each of the event. 
The plots suggest longer survival time to event 1 in the irradiated group but 
little difference between treatment groups in OC deaths.
\\
  
    
 
 \noindent \textit{Table \ref{tableex2}}\\
 
 Table \ref{tableex2} compares the $\HR$, its $95\%$ CI and p-value for the CS 
and SD proportional hazards models. 
The CI is similar  for the CS and SD hazard models. 
. 
In these two models, the treatment difference is not significant but near the level of statistical significance $\alpha=$0.05: $p=0.098$ for CS, $p=0.085$ for SD 
model.
Note the small estimates of hazard ratio of relapse cancer death 
in XRT patientsrelative to patients receiving surgery alone. 
\\


The BCR model  using EM estimator of the correlation doesn't converge due to the 
small numbers(n=3) of events in the irradiated group. 

\noindent \textit{Table \ref{tables2}}\\
\noindent \textit{Figure \ref{exsurvs2}}\\   
\noindent \textit{Figure \ref{exhazard}}\\  

In table \ref{tables2}, we therefore perform a sensitivity analysis imputing the censored 
survival time of the outcome for different levels of correlation. 
Figure \ref{exhazard} illustrates the $\MR$ and its CI estimated by the different 
methods. 
The treatment difference is significant for all values of $\rho_{\BCR}$ in favor 
of the irradiated group. 

The ML estimates of the relative increase increases as $\rho_{\BCR}$ increases 
and is close to the SD estimate for $\rho_{\BCR}=0$. 
The events may be not correlated. 

Difficulties occur with small sample size and few events of interest.\\ 

Figure \ref{exsurvs2} illustrates the cumulative incidence functions for 
different specified correlations $\rho_{\BCR}$ with a truncation time at 20 
years. 
The prediction clearly show a difference in the survival estimates between the 
two arms with a larger hazard for the non-irradiated group for different level 
of correlation. 
Additional results, in supplementary materials, show the ML estimates of the 
relative increase can be sensitive to the choice of the transformation $x^*$, 
where, for different transformations, the correlation level seems small.\\
     
In supplementary material, we have modified the data, adding some events, the 
estimates suggest a moderate correlation.
Code is available on request.

\section{Conclusion}
\label{summary}
In this paper, we addressed the estimation of the correlation between two 
competing risks and the impact on survival analysis. 
We have provided an approach to help in the understanding and choice of the time 
to event methods. 
Moreover, our simulations have shown the BCR model gives good estimations of 
means, {relative increase}, and variances but can underestimate  correlation. 
This finding suggests the need to conduct sensitivity analysis. 
In other work, our simulation has explored the influence of correlation  on 
estimation of the hazard ratio by standard methods; 
the Fine and Gray approach appears to remain robust under moderately correlated risks. 

Sensitivity analysis shows the impact of the chosen correlation on the results. 
Maximum 
likelihood estimation estimation can lack accuracy.
In such a circumstance, sensitivity analysis is indicated. 
Comparing the different methods can help to determine an upper bound of the 
potential correlation that can be used in prediction. This work is useful when 
we want to compare correlations between different competing risks. 

We have also applied our methodology to a real data set in the field of cancer 
disease. This work has use in clinical interpretation of results.

Finally, we provided 
a method to assist estimation of correlation between risks which may be useful 
in designing later studies.\\

A limitation of this work is the bivariate normal distribution assumption. 
Extending our work to semi parametric estimation where the distribution of the 
error term is not specified would be valuable. 
Such extension could employ the Buckley-James method or multiple imputation.\\
   
%We can note a competing risk imputation could be considered unrealistic but 
%this estimation can be useful for prediction  as far as the sensitivity of the 
%correlation and the need to choice the right correlation.\\
% {We can appy this work on other distributions. 
% We can use the copula function to generate correlated risks. 
% We can use optimize by R function.} \cite{Hofert2012}

Other models still deserve some attention in future work. 

 The bivariate exponential model \cite{Marshall1967a} allows simultaneous causes 
of death that might be appropriate to access of the correlation between 
competing risks. 
Investigating the maximum likelihood estimator performance and the effects of 
the estimated correlation on the {relative increase} estimation also constitute 
an interesting topic for future research.
                                                                                 
%\subsection{Second level heading}
                                                                                 
%This is the body text. Please note that cross-references in the body text should be shown as follows:
%(Miller, 1900), (Miller and Baker, 1900) or if three or more authors (Miller {\it{et al}}., 1900)
%\vspace*{12pt}
                                                                                 
%\noindent Bullet lists are not allowed. Always use (i), (ii), etc.
%\vspace*{12pt}
                                                                                 
%\noindent Sentences should never start with a symbol.
%\vspace*{12pt}
                   
%\noindent Names of software packages and website addresses should be written in {\tt{Courier new, i.e. Stata, the R package
%MASS, http://www.biometrical-journal.com.}}

\nocite{*}

\begin{table}[htb]
\begin{center}
\caption{The caption of a table.}
\begin{tabular}{lll}
\hline
Description 1 & Description 2 & Description 3\\
\hline
Row 1, Col 1 & Row 1, Col 2 & Row 1, Col 3\\
Row 2, Col 1 & Row 2, Col 2 & Row 2, Col 3\\
\hline
\end{tabular}
\end{center}
\end{table}
\begin{equation}
\left({\theta^{0}_{i}}\atop{\theta^{1}_{i}}\right) \sim N(\theta,\Sigma),\quad {\mathrm{with}}\ 
{{\theta}} = \left({\theta_{0}}\atop{\theta_{1}}\right)\ {\mathrm{and}}\ \Sigma =
\left(\begin{array}{cc}
\sigma^{2}_{0} & \rho\sigma_{0}\sigma_{1}\\
\rho\sigma_{0}\sigma_{2} & \sigma^{2}_{1}
\end{array}\right).
\end{equation}

\noindent This is the body text. Only number equations which are referred to in the text body. If equations
are numbered, these should be numbered continuously throughout the text. Not section wise! Please
carefully follow the rules for mathematical expressions in the ``Instructions to Authors''.

\begin{figure}[htb]
\begin{center}
\includegraphics[bb= 0 0 115 87]{empty.eps}
\caption{The figure caption ($b_{2}b_{2}a^{n}a^{n}$).}
\end{center}
\end{figure}
\begin{acknowledgement}
An acknowledgement may be placed at the end of the article.
\end{acknowledgement}
\vspace*{1pc}

\noindent {\bf{Conflict of Interest}}

\noindent {\it{The authors have declared no conflict of interest. (or please state any conflicts of interest)}}

 \newpage 
 \appendix
 %\section{Appendix}\label{annex}
 
 \input{probabilitySteinIdentities.tex} 
 \input{impute.tex}
% \input{StdErrors.tex}
% \input{appendixSteinProofs.tex}
% \input{AppendixSimulations.tex}
\end{document}



%Please insert appendices before the references.

\begin{thebibliography}{10}
\bibitem[Bauer and Bauer(1994)Bauer, P. and Bauer, M.M.]{bib1}Bauer, P. and  Bauer, M. M. (1994). Testing equivalence simultaneously for location and  dispersion of two normally distributed populations.  \textit{Biometrical  Journal} \textbf{36}, 643--660.
\bibitem[Farrington, C. P. and Andrews, N. (2003)]{bib2}Farrington, C.P. and Andrews, N. (2003). Outbreak detection:
Application to infectious disease surveillance. In: Monitoring the Health of Populations (eds. R. Brookmeyer and D. F. Stroup), Oxford University Press, Oxford,\break 203--231.
\bibitem[Rencher(1998)Rencher, A.C.]{bib3}Rencher, A. C. (1998).  \textit{Multivariate Statistical Inference and Applications}. Wiley, New  York. 
\end{thebibliography}
\newpage
\phantom{aaaa}

\nocite{*}

\bibliographystyle{chicago} 
%\bibliography{BNCM}
\bibliography{~/repos/References/bib/BNCM}
\end{document}